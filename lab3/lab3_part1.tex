\documentclass{article}

\usepackage[margin=1in]{geometry}
\usepackage{amsmath,amssymb}
\usepackage{parskip}

\usepackage{subcaption}


\def\*#1{\boldsymbol{#1}}

\def\ci{\perp\!\!\!\perp}
\DeclareMathOperator*{\argmax}{arg\,max}

\begin{document}

\title{ML2 - Homework 4}
\author{By Joost van Amersfoort and Otto Fabius}
\maketitle

\section*{1.}

$Q$ approximates $p(\Theta|D)$, this can be seen from the input to the KL divergence, as given in equation 11 in Bishop99.

\section*{2.}

\begin{align*}
\mathcal{L}(Q) &= \int Q(\*\theta) \ln \frac{P(D, \*\theta)}{Q(\*\theta)}d\*\theta \\
Q(\*\theta) &= \prod_k Q_k(\theta_k)
\end{align*}

Substitute in Q, write out the natural logarithm, split the integral, pull $Q_i(\theta_i)$ out of the product and rewrite as the rest as an expectation:

\begin{align*}
\mathcal{L}(Q) &= \int \prod_k Q_k(\theta_k) \ln \frac{P(D, \*\theta)}{\prod_k Q_k(\theta_k)}d\*\theta \\
&= \int \prod_k Q_k(\theta_k) \left [ \ln P(D, \*\theta) - \sum_k \ln Q_k(\theta_k) \right ] d\*\theta \\
&= \int Q_i(\theta_i) \left [ \int \ln P(D, \*\theta) \prod_{k \neq i} Q_k(\theta_k) d\theta_k \right ] d\theta_i - \int Q_i(\theta_i) \ln Q_i(\theta_i) d\theta_i \\
&= \int Q_i(\theta_i) \langle \ln P(D, \*\theta) \rangle_{k \neq i} d\theta_i - \int Q_i(\theta_i) \ln Q_i(\theta_i) d\theta_i
\end{align*}

Now note that the first term is a negative KLD between $Q_i(\theta_i)$ and the expectation, which means that maximizing this expression means minimizing this KLD. We know that the KLD is minimized when $Q_i(\theta_i)$ equals the expectation:

\begin{align*}
\ln Q_i(\theta_i) & \propto \langle \ln P(D, \*\theta) \rangle_{k \neq i} \\
Q_i(\theta_i) &\propto \exp (\langle \ln P(D, \*\theta) \rangle_{k \neq i}) \\
\end{align*}

Now we still need to normalize this, giving:

\begin{align*}
Q_i(\theta_i) &= \frac{\exp (\langle \ln P(D, \*\theta) \rangle_{k \neq i})}{\int \exp (\langle \ln P(D, \*\theta) \rangle_{k \neq i}) d\theta_i} \\
\end{align*}

\subsection*{3.}

No, we cannot use this to assess the convergence of the VB PCA algorithm, because we cannot compute the true posterior which is necessary to evaluate the joint. We only have a lowerbound, computed with Q which approximates the posterior.



\end{document}
